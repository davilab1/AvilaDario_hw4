\documentclass[reprint,amsmath,amssymb]{revtex4-1}

\usepackage[spanish, activeacute]{babel}
\usepackage[utf8]{inputenc}
\usepackage{graphicx}
\graphicspath{ {images/} }
\usepackage{enumerate}
\usepackage{hyperref}

\usepackage{color}
\usepackage{fancyhdr}

\usepackage{textcomp}

\begin{document}

\title{\Huge Tarea 4 Metodos Computacionales 2018-II }

\author{Dario Alejandro Ávila }
\email{da.avilab@uniandes.edu.co, 201630564}


\affiliation{Universidad de los Andes \\}


\maketitle
\section{\label{sec:level1}ODE}

\begin{equation}

\end{equation}

\begin{figure}[h!]
    \centering
    \includegraphics[scale=0.22]{1.jpg}
    \caption{Trayectoria del proyectil}

\end{figure}

\begin{figure}[h!]
    \centering
    \includegraphics[scale=0.22]{2.jpg}
    \caption{Trayectorias del proyectil en funcion del angulo}
\end{figure}


\section{\label{sec:level1}PDE}
\subsection{Caso 1}



\begin{figure}[h!]
    \centering
    \includegraphics[scale=0.22]{3.jpg}
    \caption{Caso 1, condiciones iniciales}
\end{figure}

\begin{figure}[h!]
    \centering
    \includegraphics[scale=0.22]{3.jpg}
    \caption{Caso 1, condiciones intermedias}
\end{figure}

\begin{figure}[h!]
    \centering
    \includegraphics[scale=0.22]{3.jpg}
    \caption{Caso 1, configuracion de equilibrio}
\end{figure}

\begin{figure}[h!]
    \centering
    \includegraphics[scale=0.22]{3.jpg}
    \caption{Caso 1, temperatura promedio de la calcita en funcion del tiempo}
\end{figure}


\subsection{Caso 2}

\begin{figure}[h!]
    \centering
    \includegraphics[scale=0.22]{3.jpg}
    \caption{Caso 2, condiciones iniciales}
\end{figure}

\begin{figure}[h!]
    \centering
    \includegraphics[scale=0.22]{3.jpg}
    \caption{Caso 2, condiciones intermedias}
\end{figure}

\begin{figure}[h!]
    \centering
    \includegraphics[scale=0.22]{3.jpg}
    \caption{Caso 2, configuracion de equilibrio}
\end{figure}

\begin{figure}[h!]
    \centering
    \includegraphics[scale=0.22]{3.jpg}
    \caption{Caso 2, temperatura promedio de la calcita en funcion del tiempo}
\end{figure}

\subsection{Caso 3}


\begin{figure}[h!]
    \centering
    \includegraphics[scale=0.22]{3.jpg}
    \caption{Caso 3, condiciones iniciales}
\end{figure}

\begin{figure}[h!]
    \centering
    \includegraphics[scale=0.22]{3.jpg}
    \caption{Caso 3, condiciones intermedias}
\end{figure}

\begin{figure}[h!]
    \centering
    \includegraphics[scale=0.22]{3.jpg}
    \caption{Caso 3, configuracion de equilibrio}
\end{figure}

\begin{figure}[h!]
    \centering
    \includegraphics[scale=0.22]{3.jpg}
    \caption{Caso 3, temperatura promedio de la calcita en funcion del tiempo}
\end{figure}



\end{document}
