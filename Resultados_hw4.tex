\documentclass[reprint,amsmath,amssymb]{revtex4-1}

\usepackage[spanish, activeacute]{babel}
\usepackage[utf8]{inputenc}
\usepackage{graphicx}
\graphicspath{ {images/} }
\usepackage{enumerate}
\usepackage{hyperref}

\usepackage{color}
\usepackage{fancyhdr}

\usepackage{textcomp}

\begin{document}

\title{\Huge Tarea 4 Metodos Computacionales 2018-II }

\author{Dario Alejandro Ávila }
\email{da.avilab@uniandes.edu.co, 201630564}


\affiliation{Universidad de los Andes \\}


\maketitle
\section{\label{sec:level1}ODE}

\begin{equation}

\end{equation}

\begin{figure}[h!]
    \centering
    \includegraphics[scale=0.22]{posxtiempo.pdf}
    \caption{Trayectoria del proyectil}

\end{figure}

\begin{figure}[h!]
    \centering
    \includegraphics[scale=0.22]{posxtiempo2.pdf}
    \caption{Trayectorias del proyectil en funcion del angulo}
\end{figure}

Como es posible observar en las dos graficas que se acaban de presentar , en la figura 1 se puede apreciar la trayectoria del proyectil para 45 grados. Por otra parte a la hora de variar los angulos de este, es posible notar que el mayor angulo que presenta una mayor extension en x es el angulo de 20 grados. Generalmente en un tiro parabolico, el major angulo es el de 45 grados, en este caso se da que sea un valor menos debido a que la friccion tiene una magnitud de v cuadrado, por lo tanto, a mayor angulo, mayor altura, el alcance sera menor. Asi, tampoco con un angulo de 10 grados es posible notar un alto alcance, el angulo ideal en este caso especifico es de 20 grados con un alcance maximo en x de 4.90

\section{\label{sec:level1}PDE}
\subsection{Caso 1}



\begin{figure}[h!]
    \centering
    \includegraphics[scale=0.22]{c1g1inial.pdf}
    \caption{Caso 1, condiciones iniciales}
\end{figure}

\begin{figure}[h!]
    \centering
    \includegraphics[scale=0.22]{c1g2inter.pdf}
    \caption{Caso 1, condiciones intermedias}
\end{figure}

\begin{figure}[h!]
    \centering
    \includegraphics[scale=0.22]{c1g3inter.pdf}
    \caption{Caso 1, configuracion de equilibrio}
\end{figure}

\begin{figure}[h!]
    \centering
    \includegraphics[scale=0.22]{c1g4equi.pdf}
    \caption{Caso 1, temperatura promedio de la calcita en funcion del tiempo}
\end{figure}

De este modo, al observar las graficas del caso. Es posible ver como las condiciones iniciales definen el sistema. Se ven unas fronteras fijas a 10 grados y la varilla a 100 grados en la mitad de la lamina de la roca. A medida que se aumenta la cantidad de interaciones, es decir, el "tiempo" se puede ver la transicion entre las graficas de como se va dando un balance en la temperatura, donde la varilla permanece fija a 100 grados y posteriormente comienza a calentar la zona alrededor. Esta zona comienza a aumentar de tempratura y es lo que se puede apreciar en las graficas.
%\begin{figure}[h!]
%    \centering
%    \includegraphics[scale=0.22]{c1g5prom.pdf}
%    \caption{Caso 1, temperatura promedio de la calcita en funcion del tiempo}
%\end{figure}

\subsection{Caso 2}

\begin{figure}[h!]
    \centering
    \includegraphics[scale=0.22]{c2g1inial.pdf}
    \caption{Caso 2, condiciones iniciales}
\end{figure}

\begin{figure}[h!]
    \centering
    \includegraphics[scale=0.22]{c2g2inter.pdf}
    \caption{Caso 2, condiciones intermedias}
\end{figure}

\begin{figure}[h!]
    \centering
    \includegraphics[scale=0.22]{c2g3inter.pdf}
    \caption{Caso 2, configuracion de equilibrio}
\end{figure}

\begin{figure}[h!]
    \centering
    \includegraphics[scale=0.22]{c2g4equi.pdf}
    \caption{Caso 2, temperatura promedio de la calcita en funcion del tiempo}
\end{figure}

\begin{figure}[h!]
    \centering
    \includegraphics[scale=0.22]{c2g5prom.pdf}
    \caption{Caso 2, temperatura promedio de la calcita en funcion del tiempo}
\end{figure}
En este caso es posible notar como hay una variacion en la temperatura de la lamina, pese a que se encuentra con la barra a 100 grados, hay una variacion que se deberia comportar similar a las condiciones periodicas, puesto que el borde se comportara como se comporta la posicion anterior a esta. De este modo, se va a dar una regulacion de temperatura gradual hasta el punto en el que toda la lamina se encuentre a 100 grados, la difcultad de lograr ver esta transicion es que es necesario una cantidad demasiado grande de iteraciones, las cuales en este caso, el computador no es capaz de soportar.
\subsection{Caso 3}

\begin{figure}[h!]
    \centering
    \includegraphics[scale=0.22]{c3g1inial.pdf}
    \caption{Caso 3, condiciones iniciales}
\end{figure}

\begin{figure}[h!]
    \centering
    \includegraphics[scale=0.22]{c3g2inter.pdf}
    \caption{Caso 3, condiciones intermedias}
\end{figure}

\begin{figure}[h!]
    \centering
    \includegraphics[scale=0.22]{c3g3inter.pdf}
    \caption{Caso 3, configuracion de equilibrio}
\end{figure}

\begin{figure}[h!]
    \centering
    \includegraphics[scale=0.22]{c3g4equi.pdf}
    \caption{Caso 3, temperatura promedio de la calcita en funcion del tiempo}
\end{figure}

%\begin{figure}[h!]
%    \centering
%    \includegraphics[scale=0.22]{c3g5prom.pdf}
%    \caption{Caso 3, temperatura promedio de la calcita en funcion del tiempo}
%\end{figure}
En este caso, es posible notar que no hay presencia de una grand diferencia entre el caso 2 y 3. Esto es posible a que se da una simetria en el sistema. Lo que sucede a un lado, depende completamente del otro lado. De este modo, al variar un lado  hara que varie la temperatura del mismo modo como si esta estuviera 

\end{document}
